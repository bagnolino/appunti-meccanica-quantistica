% Lingua e microtipografia
\usepackage[italian]{babel}
\usepackage[T1]{fontenc}
\usepackage[utf8]{inputenc}
\usepackage{microtype}

% Matematica
\usepackage{amsmath, amssymb, amsthm, mathtools}
\usepackage{siunitx}

% Grafica
\usepackage{graphicx}
\graphicspath{{figures/}}
\usepackage{tikz}
\usepackage{pgfplots}
\pgfplotsset{compat=1.18}

% Riferimenti intelligenti e link
\usepackage[hidelinks]{hyperref}
\usepackage[nameinlink,noabbrev]{cleveref}

% Bibliografia (biber + biblatex)
\usepackage{csquotes}
\usepackage[backend=biber,style=alphabetic,maxbibnames=9]{biblatex}
\addbibresource{biblio/references.bib}

% Elenchi
\usepackage{enumitem}

% Teoremi (sposta in sty/ambiente-teoremi.sty se vuoi)
\newtheorem{theorem}{Teorema}[section]
\newtheorem{lemma}[theorem]{Lemma}
\theoremstyle{definition}
\newtheorem{definition}[theorem]{Definizione}
\theoremstyle{remark}
\newtheorem{remark}[theorem]{Osservazione}

% Comandi utili
\newcommand{\R}{\mathbb{R}}
\setcounter{secnumdepth}{3}
\setcounter{tocdepth}{2}