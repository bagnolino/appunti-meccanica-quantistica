% 2 Lezione del 30/09/2025


\chapter{Postulati della Meccanica Quantistica}


\section{Stati in MQ}

\subsection{Stati classici}
Iniziamo considerando gli stati in meccanica classica.
Nel formalismo hamiltoniano, uno stato all'istante \(t\) 
è determinato da: \(\left\{ q_i(t), p_i(t) \right\}\quad i = 1,\dots, N\) .\\
Dati \(\left\{ q_i(t_0),p_i(t_0) \right\}\), l'evoluzione temporale \(\left\{ q_i(t), p_i(t)\right\}\) è determinata dalle equazioni di Hamilton:
\begin{equation}
    \dot{q}_i= \pdv{H}{p_i } \quad \dot{p}_i = -\pdv{H}{q_i }
\end{equation}

Le osservabili sono descritte da funzioni \(f(q_i(t),p_i(t),t)\) la cui dipendenza del tempo è data:
\[
\dv{t} f = \sum_{i=1}^{N} \left( \pdv{f}{q_i}\dot{q}_i+ \pdv{f}{p_i}\dot{p}_i  \right)  +\pdv{f}{t}
= \sum_{i=1}^{N} \left( \pdv{f}{q_i}\pdv{H}{p_i}-\pdv{f}{p_i}\pdv{H}{q_i} \right)+\pdv{f}{t}
= \left\{ f, H \right\} +\pdv{f}{t}
\]

Dove nell'ultimo passaggio abbiamo definito le \textit{parentesi di Poisson}:
\begin{equation}
    \left\{ f,g \right\}:= \sum_{i=1}^{N}\left( \pdv{f}{q_i}\pdv{g}{p_i}-\pdv{f}{p_i}\pdv{g}{q_i} \right)    
\end{equation}

\begin{remark}
    Dato \(q(t), p(t)\) le osservabili $f$ sono ben definite.
\end{remark}


\subsection{Stati quantistici}
In meccanica quantistica, lo stato \(\Sigma\) da solo informazione sulla distribuzione di probabilità.

\begin{definition}
    Definiamo \textit{stato puro}, lo stato che contiene tutta l'informazione disponibile su un sistema.
    Altrimenti si parla di \textit{stati misti}.
\end{definition}
Per ora ci limiteremo agli stati puri.

\begin{example}
    \textbf{Particella senza spin in 1-dim inifinita}\\
    Come stato \(\Sigma\) in questo caso consideriamo la funzione d'onda \(\Psi(x)\) . \\
    La probabilità di trovare la particella tra  \(x\)  e  \(x+\dd{x}\)  è  \(\dd{P_\Sigma}(x)= \rho_\Sigma(x)\dd{x}\,\), 
    per la quale vale la \textbf{legge di Born}: \(\rho_\Sigma\propto \abs{\Psi(x)}^2\) .

    Essendo \(\rho_\Sigma\) una densità di probabilità deve valere: \(\rho_\Sigma \geq 0 \) e \(\int_\mathbb{R} \rho_\Sigma(x) \dd{x}= 1\), otteniamo dunque:
    \begin{equation}
        \rho_\Sigma(x) = \frac{\abs{\Psi(x)}^2}{\norm{\Psi}^2} 
    \end{equation}
    dove abbiamo definito \(\norm{\Psi}^2 = \int_\mathbb{R}\abs{\Psi}^2 \dd{x}\) .

    Questa condizione pone dei limiti rilevanti al tipo di funzioni a cui può appartenere \(\Psi(x)\):
    \begin{enumerate}
        \item Non può essere identicamente nulla: \(\Psi(x)= 0 \;\forall \, x\) .
        \item Deve essere integrabile: \(\int_\mathbb{R} \abs{\Psi(x)}^2 \dd{x} < + \infty \implies \Psi(x) \in L_2(\mathbb{R}, \dd{x})\)
        \item Funzioni d'onda che differiscono per uno scalare rappresentano la stessa densità di probabilità.
    \end{enumerate}
    
    \begin{remark}
        \(\Psi(x)\) è un vettore in uno spazio di Hilbert.
    \end{remark}
\end{example}


\subsection{Richiami su spazi di Hilbert}
Si dice \textit{spazio di Hilbert} \(\mathcal{H}\), uno spazio vettoriale su \(\mathbb{C}\) dotato di prodotto scalare e completo, 
che ha le seguenti proprietà \(\forall \,\psi,\chi\):
\begin{itemize}
    \item \( \ip{\psi}{\chi} \in \mathbb{C}\)
    \item \(\ip{\psi}{a\chi_1+c\chi_2}= a\ip{\psi}{\chi_1}+b\ip{\psi}{\chi_2}\)
    \item \(\ip{\psi}{\chi}= \overline{\ip{\chi}{\psi}}\)
    \item \(\ip{\psi}\geq 0\quad \forall \,\psi\)
\end{itemize}

Definiamo inoltre la \textit{norma} \(\norm{\psi}: = \sqrt{\ip{\psi}{\psi} }\)

\begin{example}
    Sono spazi di Hilbert:
    \begin{itemize}
        \item \(\mathcal{H}= L_2(\mathbb{R},\dd{x})\) con prodotto scalare \(\ip{\psi}{\chi}= \int_{\mathbb{R}} \overline{\psi(x)}\chi(x)\dd{x}\)
        \item \(\mathcal{H}= \mathbb{C}^n\) con prodotto scalare \(\ip{u}{v}= \sum_{k=1}^{n} u_k^* v_k\)
    \end{itemize}
\end{example}

Per un qualsiasi spazio di Hilbert \(\mathcal{H}\) vale la \textbf{disuguaglianza di Schwarz}:
\begin{equation}
    \abs{\ip{\psi}{\chi}}\leq \norm{\psi}\norm{\chi}
\end{equation}

Inoltre definiamo:
\begin{definition}
    Uno spazio di Hilbert \(\mathcal{H}\) si dice \textit{separabile} se e solo se \(\exists \, S \subseteq \mathcal{H}\) numerabile e \(\overline{S}= \mathcal{H}\).
\end{definition}
\begin{definition}
    Per ogni spazio di Hilbert \(\mathcal{H}\) si definisce lo \textit{spazio duale} \(\mathcal{H}^* \):
    \[
    \mathcal{H}^*= \left\{ \text{funzionali lineari }\phi: D_\phi = \mathcal{H}\to \mathbb{C} \text{ continui}\right\}
    \]
\end{definition}
\begin{definition}
    Un funzionale lineare \(\phi\) è \textit{continuo} se data una successione \(\psi_n \overset{\mathcal{H}}{\to} \psi\) si ha \(\phi(\psi_n)\to \phi(\psi)\) .
    Un funzionale è continuo se e solo se è \textit{limitato}, cioè se \(\exists\, M> 0 : \abs{\phi(\psi)} \leq M \norm{\psi}\) .
\end{definition}


\begin{example}
    Fissato \(\psi\in \mathcal{H}\) esiste un \(\phi_\psi \in \mathcal{H}^*\) dato da:
    \begin{equation}
        \xi\mapsto \phi_\psi(\chi):= \ip{\psi}{\chi}
    \end{equation}
    Si dimostra per il \textbf{teorema di rappresentazione di Riesz} che tutti i \(\phi \in \mathcal{H}^*\) si ottengono in questo modo.

    \begin{remark}
        Anche \(\mathcal{H}^*\) è spazio di Hilbert con prodotto scalare:
        \[
        \ip{\phi_\psi}{\phi_\chi}_{\mathcal{H}^*}:= \overline{\ip{\psi}{\chi}_\mathcal{H}}= \ip{\chi}{\psi}_\mathcal{H}
        \]
    \end{remark}
\end{example}


\subsubsection{Notazione di Dirac}
Rappresentiamo i vettori \(\psi \in \mathcal{H}\) come \textit{ket} \(\ket{\psi}\) , 
invece i funzionali \(\phi\in \mathcal{H}^*\) come \textit{bra} \(\bra{\phi}\) .
C'è una corrispondenza biunivoca di dualità tra il ket \(\ket{\psi}\in \mathcal{H}\) e il bra \(\bra{\psi} \in \mathcal{H}^*\).
Con \(\braket{\psi}{\chi}\) indichiamo il funzionale \(\bra{\psi}\) appliicato al vettore \(\ket{\chi}\)
che equivale al prodotto scalare tra i vettori \(\ket{\psi}\) e \(\ket{\chi}\) .
\begin{attention}
    I ket si scompongono linearmente, invece i bra antilinearmente:
    \[
    \ket{a\psi_1+b\psi_2}= a\ket{\psi_1}+ b\ket{\psi_2} \iff \bra{a\psi_1+b\psi_2}= a^*\bra{\psi_1}+ b^* \bra{\psi_2}
    \]
\end{attention}

\begin{example}
    \[
    \mathcal{H}= \mathbb{C}^n   \quad \ket{u}= \begin{pmatrix}
        u_1\\\vdots \\ u_n  
    \end{pmatrix}
    \quad \bra{u}= \begin{pmatrix}
        u_1 & \dots & u_n
    \end{pmatrix}
    \]
\end{example}



\begin{theorem}
    \(\mathcal{H} \) spazio di Hilbert separabile \(\implies  \;\exists\,\) base ortonormale
    \(\left\{ \ket{n} \right\}_{n \in \mathbb{N}}\)
\end{theorem}
Inoltre per \(\forall \ket{\psi} \in \mathcal{H}\) è possibile la scomposizione:
\(\ket{\psi}= \sum_{n=1}^{\infty}c_n\ket{n}\) con \(c_n = \braket{n}{\psi}\) .
Da cui notiamo \(\ket{\psi}= \sum_{n=1}^{\infty}\ket{n}\braket{n}{\psi} \implies \mathbb{I}_n= \sum_{i=1}^{\infty} \dyad{n}{n}\)
detta \textbf{relazione di completezza}.

\begin{example}
    \[
    \braket{\psi}{\chi}= \matrixel{\psi}{\mathbb{I}}{\chi}=
    \sum_{n=1}^{\infty} \braket{\psi}{n}\braket{n}{\chi}  = \sum_{n=1}^{\infty} a^*_n b_n
    \]
    dove si è considerato \(\ket{\psi}= \sum_{n=1}^{\infty}a_n \ket{n} \,, 
    \, \ket{\chi}= \sum_{n=1}^{\infty}b_n \ket{n}\)
\end{example}

\begin{example}
    \[
    \norm{\psi}^2= \braket{\psi}= \sum_{n=1}^{\infty} \braket{\psi}{n}\braket{n}{\psi}=
    \sum_{n=1}^{\infty}c^*_n c_n    = \sum_{n=1}^{\infty} \abs{c_n}^2   
    \]
\end{example}


\subsection{Rappresentazione x e p}
Consideriamo una funzione d'onda \(\psi(x) \in L_2(\mathbb{R}, \dd{x})\),
la decompongo in onde monocromatiche \(e^{ikx}\) e considero la \textbf{legge di Broglie}: \(\frac{2\pi}{\lambda}= k = \frac{p}{\hbar}\), otteniamo:
\[
\psi(x)= \frac{1}{\sqrt{2\pi\hbar}}\int_{\mathbb{R}} \dd{p} e^{i\frac{px}{\hbar}} \tilde{\psi}(p) = \mathcal{F}^{-1}[\tilde{\psi}(p)](x)
\]
\[
\tilde{\psi}(p)= \frac{1}{\sqrt{2\pi\hbar}}\int_{\mathbb{R}} \dd{x} e^{-i\frac{px}{\hbar}} \psi(x)= \mathcal{F}[\psi(x)](p)
\]
La densità di probabilità tra \(p\) e \(p +\dd{p}\) è \(\rho_\Sigma= \frac{\abs{\tilde{\psi}(p)}^2}{\norm{\tilde{\psi}}^2}\).
\begin{theorem}
    \textbf{Plancherel}
    \[
    \norm{\tilde{\psi}}^2= \int_{\mathbb{R}} \dd{p} \abs{\tilde{\psi}(p)}^2= \int_{\mathbb{R}} \dd{x} \abs{\psi(x)}^2 = \norm{\psi}^2 <\infty
    \]
\end{theorem}
Dunque, \(\tilde{\psi}(p)\) appartiene a \(L_2(\mathbb{R},\dd{p})\) .

\begin{definition}
    \(\psi(x)\) si dice \textit{rappresentazione} \(x\) dello stato, mentre
    \(\tilde{\psi}(p)\) si dice \textit{rappresentazione} \(p\) dello stato.
\end{definition}

\begin{theorem}
    Tutti gli spazi di Hilbert della stessa dimensione sono isomorfi tra di loro.
\end{theorem}

\begin{example}
    Se \(\operatorname{dim} \mathcal{H} < \infty \implies \mathcal{H}\simeq \mathbb{C}^n\),invece
    se \(\operatorname{dim} \mathcal{H} = \infty \implies \mathcal{H} \simeq \ell_2\) .\\
    Dato l'isomorfismo \(L_2(\mathbb{R},\dd{x})\simeq \ell_2\) posso rappresentare una funzione \(\psi(x)\) come un vettore \(\ket{\psi}\in \ell_2\) .
\end{example}

\begin{definition}
    Definisco il \textit{raggio vettore} \([\Psi]\) come l'insieme dei vettori che rappresentano lo stesso stato:
    \[
    [\Psi ]= \big\{ \alpha\ket{\Psi}, \alpha \in \mathbb{C} \setminus \left\{ 0 \right\} \big\}
    \]
\end{definition}

\begin{postulato1}
    Ogni sistema quantistico \(S\) può essere associato a uno spazio di Hilbert \(\mathcal{H}\), 
    mentre ogni stato puro \(\Sigma \in S\) è associato a un raggio vettore \([\Psi ]\in \mathcal{H}\) .
\end{postulato1}

%3 Lezione del 02/10/2025

\begin{example}
    \(\mathcal{H}= \mathbb{C}^3 \quad e_1= \begin{pmatrix}
        1   & 0 & 0
    \end{pmatrix}\quad  \Psi_1= (e_2-e_1)\,,\; \Psi_2= \frac{1}{\sqrt{2}}\left( e_1-e_2 \right)\,,\; \Psi_3 = i(e_1+e_2)\)
    tra queste tre funzioni d'onda \(\Psi_1 \) e \( \Psi_2 \) rappresentano lo stesso stato.
\end{example}


\subsection{Basi generalizzate}

Consideriamo il nostro spazio di Hilbert \(\mathcal{H} = L_2(\mathbb{R}, \dd{x})\), 
possiamo quindi rappresentare le nostra funzione d'onda \(\Psi(x)\) come vettore \(\ket{\Psi}\); 
vogliamo trovare una base ortonormale \(\left\{ \ket{n} \right\}_{n \in \mathbb{N}} \equiv \left\{ e_n(x) \right\}_{n \in \mathbb{N}}\) 
per la quale valga:
\[
    \ket{\Psi}= \sum_{n= 1}^{\infty} \ket{n}\braket{n}{\Psi} \quad \Psi(x)= \sum_{n=1}^{\infty}c_n e_n(x)
    \quad c_n = \braket{n}{\Psi}= \int_{\mathbb{R}} \overline{e_n}(x)\Psi(x)\dd{x}
\]

La scomposizione in onde monocromatiche è una generalizzazione di questo concetto,
possiamo infatti considerare la \textit{base generalizzata} \(\left\{ \ket{p} \right\}_{p \in \mathbb{R}}\equiv \left\{ e_p(x) = \frac{1}{\sqrt{2\pi\hbar}} e^{i\frac{px}{\hbar}} \right\}_{p \in \mathbb{R}}\) ,
per la quale vale:
\begin{equation*}
    \Psi(x)= \int_{\mathbb{R}} \dd{p} \frac{1}{\sqrt{2\pi\hbar}} e^{i\frac{p}{\hbar}x}\tilde{\Psi}(p) \quad 
    \tilde{\Psi}(p)= \braket{p}{\Psi} = \int_\mathbb{R} \dd{x} \frac{1}{\sqrt{2\pi\hbar}} e^{-i\frac{p}{\hbar}x}\Psi(x)
\end{equation*}
\begin{equation}
    \implies \ket{\Psi}= \int_\mathbb{R} \ket{p} \braket{p}{\Psi}
\end{equation}

\begin{attention}
    \(\ket{p} \notin L_2(\mathbb{R},\dd{x})\), i \(\ket{p}\) non possono rappresentare stati fisici del sistema.
\end{attention}

Si chiede, però,  che \(\ket{p} \in S^*(\mathbb{R})\).\\
Se invece di basi a momento fissato, scelgo una base a posizione fissa del tipo 
\(\left\{ \ket{x_0} \right\}_{x_0 \in\mathbb{R}}\equiv \left\{\delta_{x_0}(x)= \delta(x-x_0)\right\}_{x_0 \in \mathbb{R}}\):

\begin{equation*}
    \ket{\Psi}= \int_\mathbb{R}\dd{x_0} \ket{x_0}\braket{x_0}{\Psi} \quad   
    \braket{x_0}{\Psi} = \int_\mathbb{R} \overline{\delta_{x_0}}(x) \Psi(x)= \Psi(x_0)
\end{equation*}
\begin{equation}
    \implies \Psi(x)= \int_\mathbb{R} \dd{x_0} \delta_{x_0}(x)\Psi(x_0)
\end{equation}

Vale, inoltre, la \textit{relazione di completezza generalizzata}: 
\begin{equation}
    \int \dd{x_0} \dyad{x_0}{x_0}= \mathbf{1}_\mathcal{H} = \int \dd{p} \dyad{p}{p}
\end{equation}

Le funzioni d'onda sono i coefficienti di \(\ket{\Psi}\) rispetto a una particolare base:
\begin{equation}
    \Psi(x)= \braket{x}{\Psi} \qquad \tilde{\Psi}(p) = \braket{p}{\Psi}
\end{equation}


\subsubsection{Nota agli spazi delle funzioni}
Consideriamo \(S(\mathbb{R})\) spazio di Schwarz e \(S^*(\mathbb{R})\) spazio delle distribuzioni temperate,
vale \(S(\mathbb{R})\subseteq L_2(\mathbb{R})\subseteq S^*(\mathbb{R})\) .
Consideriamo una funzione di test \(\psi(x) \in S(\mathbb{R})\) e una distribuzione \(F(x) \in S^*(\mathbb{R})\):
\[
    F(x)(\psi) = \int_\mathbb{R} F(x) \psi(x) \dd{x} \quad \braket{F(x)}{\psi} = \int_\mathbb{R}\overline{F(x)} \psi(x) \dd{x} = \overline{F(x)} (\psi)
\]
\begin{attention}
    L'integrazione è puramente formale se \(F(x)\) è una distribuzione non regolare.
\end{attention}
In fisica, possiamo usare anche i ket che appartengono a \(S^*(\mathbb{R})\), per i quali vale:
\[
    \braket{\psi}{F(x)}= \overline{\braket{F(x)}{\psi}} 0 \int_\mathbb{R} F(x) \psi (x) \dd{x} = F(x) (\overline{\psi})
\]

Se \(\Psi(x) \in L_2(\mathbb{R}, \dd{x})\), \(\exists \, \Psi_n \in S(\mathbb{R})\) tale che \(\norm{\Psi-\Psi_n}_{L_2}\overset{n \to \infty}{\longrightarrow} 0\); 
si dimostra che \(\tilde{\Psi}(p)= \braket{p}{\Psi} : = \lim_{n\to \infty}\braket{p}{\Psi_n}\) converfe per quasi \(\forall p \in \mathbb{R}\)


\section{Osservabili}
\begin{definition}
    Un'\textit{osservabile} \(\mathcal{A}\) è una quantità fisica \textbf{misurabile } con operazioni eseguibili nel sistema quantistico \(S\) .
\end{definition}
Il risultato della misura di un'osservabile deve essere sempre \textbf{reale}.

\begin{definition}
    Definiamo lo \textit{spettro di } \(\mathcal{A}\): \(\sigma(\mathcal{A})\) come l'insieme dei valori ottenibili da una
    misura di \(\mathcal{A}\) al variare di tutti gli stati possibili di \(S\) .
\end{definition}

Lo spettro \(\sigma(\mathcal{A})\) può essere discreto, continuo o misto.

\begin{definition}
    Disponendo di \(N >> 1 \) copie del sistema \(S\) nello stesso stato \(\Sigma\), posso effettuare \(N\) misure di \(\mathcal{A}\) 
    ottenendo risultati \(a_1,a_2,\dots, a_N \in \sigma(\mathcal{A})\). Definisco:
    \begin{itemize}
        \item Il \textit{valor medio } di \(\mathcal{A}\) nello stato \(\Sigma\) come: \[
            \expval{\mathcal{A}}_\Sigma := \lim_{N\to \infty} \frac{a_1+\dots+a_N}{N}
        \]
        \item La \textit{fluttuazione quadratica media} di \(\mathcal{A}\) nello stato \(\Sigma\) attorno a \(\hat{a} \in \mathbb{R}\) come: \[
            \left( \Delta \mathcal{A} \right)_{\hat{a}, \Sigma}:= \sqrt{\expval{\mathcal{A}- \hat{a}}_\Sigma}= \sqrt{\lim_{N \to \infty} \frac{(a_1-\hat{a})^2+ \dots+ (a_N-\hat{a})^2}{N}} 
        \] Di solito si considera \(\hat{a}= \expval{\mathcal{A}}_\Sigma\); \(\left( \Delta\mathcal{A} \right)_{\expval{\mathcal{A}, \Sigma}}\) rappresenta l'incertezza della misura di \(\mathcal{A}\) nello stato \(\Sigma\) .
    \end{itemize}
\end{definition}

E' importante notare che nel caso discreto: \[
    \expval{\mathcal{A}}_\Sigma = \lim_{N\to \infty} \sum_{a \in \sigma(\mathcal{A})} a \frac{m(a)}{N} = \sum_{a \in \sigma(\mathcal{A})} a P_\Sigma(a)
\]
che si generalizza nel caso di uno spettro misto come:
\[
    \expval{A}_\Sigma = \sum_{a \in \sigma_d(\mathcal{A})} a P_\Sigma(a) + \int_{a \in \sigma_c(\mathcal{A}) } \dd{a} a \rho_\Sigma(a)
\]

\begin{example}
    In meccanica classica, dato un'osservabile \(f(q_i,p_i)\) per uno stato puro \(\Sigma\): 
    \(\expval{f}= f(q_i, p_i)\) e \(\left( \Delta f \right)_{(q_i,p_i) \expval{f}}=0\) .
\end{example}

\begin{example}
    In meccanica quantistica, consideriamo una particella senza spin 1-dim, il suo stato è descritto da una funzione d'onda \(\Psi(x) \in L_2(\mathbb{R}, \dd{x})\).
    Vogliamo misurare l'osservabile posizione \(\mathcal{X}\), il cui spettro \(\sigma(\mathcal{X}) = \mathbb{R}\) :
    \[
        \expval{\mathcal{X}}_{\Sigma= [\Psi]} = \int_{\sigma(\mathcal{X})= \mathbb{R}} x\rho_\Sigma(x) \dd{x} = \int_{\mathbb{R}} x \frac{\abs{\Psi(x)}^2}{\norm{\Psi}^2} \dd{x}= 
        \frac{1}{\norm{\Psi}^2}= \int_{\mathbb{R}} \Psi^*(x) x \Psi(x) \dd{x}
    \]
    Possiamo definire l'operatore posizione \(X: (X\Psi)(x)= x\Psi(x)\) e ottenere:
    \begin{equation}
        \implies \expval{\mathcal{X}}_{\Sigma= [\Psi]} = \frac{\expval{\mathcal{X}}{\Psi}}{\norm{\Psi}^2}
    \end{equation}
    Alla stessa maniera per l'osservabile momento \(\mathcal{P}\), il cui spettro \(\sigma(\mathcal{P})= \mathbb{R}\),
    possiamo definire l'operatore momento \(P: (P\Psi)(p)= p\Psi(p)\) e ottenere:
    \begin{equation}
        \expval{\mathcal{P}}_{\Sigma} = \int_{\mathbb{R}} \dd{p} \frac{\abs{\tilde{\Psi}(p)}^2}{\norm{\Psi}^2}= \frac{\expval{\mathcal{P}}{\Psi}}{\norm{\Psi}^2}
    \end{equation}

\end{example}

\begin{postulato2}
    Ad ogni osservabile \(\mathcal{A}\) di un sistema quantistico \(S\) è associato un operatore \(A: D(A) \subseteq \mathcal{H} \to \mathcal{H}\) tale che sia:
    \textbf{lineare}, \textbf{densamente definito} e \textbf{autoaggiunto}. \\
    Il valore medio di \(\mathcal{A}\) nello stato \(\Sigma = [\Psi]\) è:
    \begin{equation}
        \expval{\mathcal{A}}_\Sigma = \frac{\expval{A}{\Psi}}{\norm{\Psi}^2} =: \expval{A}_\Psi
    \end{equation}
\end{postulato2}


%4 Lezione del 06/10/2025

\begin{example}
    \textbf{Operatore momento P}\\
    Associa al ket \(\ket{\Psi}\mapsto P\ket{\Psi}\), in rappresentazione \(p\) 
    abbiamo \(\braket{p}{\Psi}= \tilde{\Psi}(p)\mapsto \matrixel{p}{P}{\Psi}= p\tilde{\Psi}(p)\).
    Invece in rappresentazione \(x \), otteniamo:
    \[
        \Psi(x) = \braket{x}{\Psi} = \int_\mathbb{R} \dd{p} \braket{x}{p} \braket{p}{\Psi}= 
        \int_\mathbb{R} \dd{p} \frac{1}{\sqrt{2\pi \hbar}} e^{i\frac{p}{\hbar}x} \tilde{\Psi}(p)= 
        \mathcal{F}^{-1}[\tilde{\Psi}(p)](x)
    \]
    \[
        (P\Psi)(x)= \matrixel{x}{P}{\Psi}=\int_\mathbb{R} \dd{p} \braket{x}{p} \bra{p}P\ket{\Psi}= 
        \int_\mathbb{R} \dd{p} \frac{1}{\sqrt{2\pi \hbar}} e^{i\frac{p}{\hbar}x}p \tilde{\Psi}(p)= 
    \]
    \[
        =\mathcal{F}^{-1}[p\tilde{\Psi}(p)](x)= -i\hbar \dv{x} \Psi(x) \implies P = -i\hbar \dv{x}
    \]
    Euivalentemente si dimostra che l'operatore \(X\) in rappresentazione \(p\): \(X = i\hbar \dv{p}\)
\end{example}


\subsection[Operatori lineari]{Operatori lineari su \(\mathcal{H}\)}

Data una coppia \((A, D_A)\), dove \(A: D_A \subseteq \mathcal{H}\to \mathcal{H}\) è un operatore lineare e \(D_A\) il suo dominio, 
vale la linearità: \(A(a\ket{\Psi_1}+ b\ket{\Psi_2})= aA\ket{\Psi_1}+bA\ket{\Psi_2}\) .
Si dice, inoltre, che è \textit{densamente definito} se \(\overline{D_A}= \mathcal{H}\) .

\begin{example}
    \(\mathcal{H}= L_2(\mathbb{R},\dd{x})\), per l'operatore posizione \((X\Psi)(x)= x\Psi(x)\) 
    non posso prendere come dominio tutto \(\mathcal{H}\), perché anche se \(\Psi(x) \in L_2(\mathbb{R},\dd{x})\) 
    non è detto che \(x\Psi(x) \in L_2(\mathbb{R}, \dd{x})\). 
    Ad esempio \(\Psi(x)= \frac{1}{\sqrt{x^2+1}} \in L_2(\mathbb{R},\dd{x})\), ma \(x\Psi(x) \notin  L_2(\mathbb{R}, \dd{x})\) .\\
    Di conseguenza scelgo come dominio \(D_x=  \left\{\Psi(x) \in L_2(\mathbb{R}, \dd{x}) : x \Psi(x) \in L_2(\mathbb{R}, \dd{x})  \right\}\) .
\end{example}

Un operatore è limitato se \(\exists M >0 \) t.c. \(\norm{A\Psi} \leq M \norm{\Psi}\quad \forall\, \Psi \in D_A\)
ed è limitato se e solo se è anche continuo, cioè se \(\forall\, \Psi_n \in D_A\to \Psi \in D_A, \ A\Psi_n \to A \Psi\) .

Se \(A\) è continuo e densamente definito allora posso estendere il dominio a tutto \(\mathcal{H}\) ed è continuo: \(A : \mathcal{H}\to \mathcal{H}\) . \\
Un operatore limitato corrispone a uno spettro di un'osservabile limitato.
Se \(\operatorname{dim}\mathcal{H} < \infty\), tutti gli operatori sono limitati. 

\begin{definition}
    Definiamo l'\textit{aggiunto} di un operatore \(A: \mathcal{H}\to \mathcal{H}\) limitato, l'operatore:
    \begin{equation}
        A^\dagger: \mathcal{H}\to \mathcal{H} \text{ limitato t.c. } 
        \forall \,\Psi, \chi \in \mathcal{H}\ \braket{\Psi}{A\chi} = \braket{A^\dagger\Psi}{\chi}
    \end{equation}
    Dato un operatore limitato, il suo aggiunto esiste ed è unico.
\end{definition}

\textbf{Proprietà aggiunto}:
\begin{itemize}
    \item \((aA+ bB)^\dagger= a^*A^\dagger + b^*B^\dagger \)
    \item \((AB)^\dagger= B^\dagger A^\dagger\)
    \item \((A^\dagger)^\dagger\)
\end{itemize}


\subsection{Rappresentazione matriciale}
E' possibile e comodo rappresentare gli operatori lineari come matrici:
data una base ortonormale \(\left\{ \ket{n}_{n \in \mathbb{N}} \right\}\),
ogni ket è possibile rappresentarlo come \(\ket{\Psi}= \sum_n \ket{n}\braket{n}{\Psi}= \sum_n c_n \ket{n}\), con l'operatore:
\[
    A\ket{\Psi}= \sum_n \ket{n} \bra{n}A\ket{\Psi}= \sum_n d_n \ket{n}
\]
\[
    d_n= \bra{n}A\ket{\Psi} = \sum_m \bra{n} A \ket{m}\bra{m}\ket{\Psi}= \sum_m A_{nm}c_m 
\]
\[
    \begin{pmatrix}
        d_1\\d_2\\ \vdots\\d_n
    \end{pmatrix}= 
    \begin{pmatrix}
        A_{11} & A_{12} & \dots & A_{1n} \\
        A_{21} & \ddots & &\vdots\\
        \vdots& &\ddots & \vdots \\
        A_{n1} & \dots & \dots & A_{nn}
    \end{pmatrix}
    \begin{pmatrix}
        c_1 \\
        c_2\\
        \vdots\\
        c_n 
    \end{pmatrix}
\]

Similmente, la si può rappresentare come :
\begin{equation}
    A = \sum_{n,m} \ket{n} \bra{n}A \ket{m} \bra{m} = \sum_{n,m} \ket{n}  A_{nm} \bra{m}    
\end{equation}

\begin{example}
    \textbf{Operatore identità}
    \[
        \mathbb{I}= \sum_{n,m} \ket{n}\bra{n} \mathbb{I} \ket{m}\bra{m}= \sum_{n,m} \ket{n}\delta_{nm}\bra{m}= \sum_n \ket{n}\bra{n}
    \]
\end{example}

\begin{example}
    \textbf{Operatori aggiunti}
    \[
        (A^\dagger)_{nm}= \braket{n}{A^\dagger m}= \overline{\braket{A^\dagger m}{n}}= \overline{\braket{m}{A n}}= A_{mn}^*
    \]
    L'aggiunto di un operatore euqivale a trasporre la matrice e cogniugare i suoi elementi.
\end{example}

\begin{definition}
    Dato \(A: D_A \to \mathcal{H}\) densamente definito illimitato, 
    esiste ed è unico l'operatore aggiunto \((A^\dagger, D_{A^\dagger})\):
    \begin{equation*}
        D_{A^\dagger} = \left\{ \chi \in \mathcal{H} :\sup_{\Psi \neq 0  \in D_A} \frac{\abs{\braket{\chi}{A\Psi}}}{\norm{\Psi}}<+\infty\right\}
    \end{equation*} 
    \begin{equation}
        \braket{\chi}{A\Psi} = \braket{A^\dagger \chi}{\Psi} \ \forall\, \Psi \in D_A, \chi \in D_{A^\dagger}
    \end{equation}
    La condizione sul dominio serve a garantirci che il funzionale lineare \(\Phi_{A,\chi}(\Psi):= \braket{\chi}{A\Psi}\) sia limitato.
\end{definition}


\subsection{Proprietà degli operatori per essere osservabili}
Ricordiamoci le tre caratteristiche che deve avere un operatore \(A\) per rappresentare un'osservabile \(\mathcal{A}\): 
\textbf{linearità}, \textbf{densamente definito} e \textbf{autoaggiunto}. Vediamole una per una:

\subsubsection{Linearità}
Data dalla necessita che \(\expval{A}_{\Psi}= \frac{\expval{A}{\Psi}}{\norm{\Psi}^2}\) 
deve essere invariante per trasformazioni \(\Psi\to \alpha \Psi\ (\alpha\neq 0)\) e deve sempre valere il principio di sovrapposizione per funzioni d'onda.

\subsubsection{Densamente definito}

Se abbiamo uno spettro \(\sigma(\mathcal{A})\) illimitato, non possiamo richiedere che \(D_A = \mathcal{H}\)\\
Può essere che il mio stato \(\ket{\Psi} \notin D_A\) e dunque \(\expval{A}_{\Psi}\) e \(\expval{\Delta A}_{\Psi}\) non essere definiti; 
è perfettamente ammissibile che ogni misura effettuata restituisca un risultato limitato, ma il cui valore medio non converga a nessun numero,
anche per distribuzioni di probabilità ben definite.\\
Che un operatore restituisca distribuzioni di probabilità ben definite è l'unica proprietà che ci interessa e questa è garantita da \(\overline{D_A}= \mathcal{H}\) .

\subsubsection{Autoaggiuntezza}
\begin{definition}
    Un operatore \((A, D_A)\), \(\overline{D_A}= \mathcal{H}\) è \textit{hermitiano} o \textit{simmetrico} sse:
    \begin{equation}
        \braket{\chi}{A\Psi}   = \braket{A\chi}{\Psi} \quad \forall\, \Psi, \chi \in D_A
    \end{equation}  
    o equivalentemente: \(A^\dagger\chi = A\chi \quad   \forall\,\chi   \in D_A\subseteq D_{A^\dagger}\)
\end{definition}
\begin{definition}
    Un operatore \((A, D_A)\), \(\overline{D_A}= \mathcal{H}\) è \textit{autoaggiunto} sse: \((A, D_A)= (A^\dagger, D_{A^\dagger})\) 
    o equivalentemente sse è hermitiano e \(D_A= D_{A^\dagger}\)
\end{definition}

\begin{remark}
    Se \(A \) è limitato allora \(D_A = \mathcal{H}= D_{A^\dagger}\) e quindi le definizioni di hermitiano e autoaggiunto equivalgono.
    Spesso in testi inglesi non si fa distinzione tra i due termini.
\end{remark}
\begin{remark}
    In rappresentazione matriciale un operatore è hermitiano sse \(A_{nm}= A_{mn}^*\) .
\end{remark}

Perché richiediamo che gli operatori siano hermitiani?
\begin{theorem}
    \begin{equation}
        \expval{A}_{\Psi}= \frac{\expval{A}{\Psi}}{\norm{\Psi}^2} \in \mathbb{R} \ \forall\, \Psi \in D_A \iff A \text{ è hermitiano.}
    \end{equation}
\end{theorem}
\begin{proof}
    Solo (\(\impliedby\)) Se \(A\) è hermitiano qualunque valor medio è reale:
    \[
        \braket{\Psi}{A\Psi}= \braket{A\Psi}{\Psi}= \overline{\braket{\Psi}{A\Psi}} \implies \braket{\Psi}{A\Psi} \in \mathbb{R}
    \]
\end{proof}

Data un'osservabile \(\mathcal{A}\), consideriamo una funzione di questa osservabile \(f: \sigma(\mathcal{A})\to\mathcal{H}\rightsquigarrow f(\mathcal{A})\) . 
Nel senso che se misuro \(\mathcal{A}\) ho ottenuto una misura anche di \(f(\mathcal{A})\), ad esempio misurando il momento ottengo una misura indiretta dell'energia cinetica.\\
Come facciamo a essere certi che questa \(f(\mathcal{A})\) a cui è associata \(f(A)\) sia ben definita?
\begin{example}\(\mathcal{A} = \mathcal{P}\)
    \[
        f(P) = P^2 = -\hbar^2\dv[2]{}{x} \ \text{ ok } \qquad f(P) = \sin(P) \ \text{ ? }
    \]
    Vedremo più avanti come comportarci in questi casi.
\end{example}

\begin{remark}
    Dato uno stato \(\Sigma\to [\Psi]\) e un'osservabile \(\mathcal{A} \to A \):
    \[
        (\Delta A )_{\hat{a}, \Sigma} = \sqrt{\expval{(\mathcal{A}-\hat{a})^2}_\Sigma}   = \sqrt{\expval{(A-\hat{a}\mathbb{I})^2}_{\Psi}}
    \]
\end{remark}

\subsection{Spettri discreti}

Vediamo ora come sono collegati gli osservabili agli operatori autoaggiunti,
ad esempio, le informazioni sullo spettro di un'osservabile devono essere contenute all'interno dell'operatore \(A\).
\begin{definition}
    \(\Sigma\) è \textit{autostato} di \(\mathcal{A}\) relativo ad \(\hat{a}\in \mathbb{R}\) sse una misura di \(\mathcal{A}\) nello stato
    \(\Sigma\) da con certezza il valore \(P_\Sigma(\hat{a})=1\) .
\end{definition}

% 5 Lezione del 08/10/2025

\begin{theorem}
    \(\Sigma\) è autostato di \(\mathcal{A}\) relativo ad \(\hat{a} \in \mathbb{R}\)   \(\iff (\Delta \mathcal{A})_{\hat{a},\Sigma}=0\)
\end{theorem}
\begin{proof}
    \((\Delta \mathcal{A})_{\hat{a}, \Sigma} = \sum_{a \in \Sigma(\mathcal{A})}(a-\hat{a})^2 P_{\Sigma}(a)\) è una somma di numeri non negativi.
    \[
        (\Delta\mathcal{A}) = 0 \iff (a- \hat{a})^2 P_\Sigma(a)= 0 \quad \forall \, a \in \sigma(\mathcal{A})
    \] 
    \[
        \iff P_\Sigma(a)= 0 \ \forall \, a \neq \hat{a}  \iff P_\Sigma(\hat{a})=1
    \]
\end{proof}

\begin{definition}  
    \textbf{Spettro discreto}
    \begin{equation}
        \sigma_d(\mathcal{A}):= \left\{ \hat{a} \in \mathbb{R} : \exists \text{ autostato } \Sigma \text{ di }\mathcal{A} \text{ relativo ad } \hat{a}\right\}
    \end{equation}
    
\end{definition}

Consideriamo ora l'operatore autoaggiunto associato \(A\) e lo stato associato \([\Psi]\):
\begin{definition}
    \(\hat{a} \in \mathbb{R}\) è \textit{autovalore} di \(A\) autoaggiunto sse \(\exists \,\Psi \neq 0 \in D_A\) tale che \((A-\hat{a})\Psi=0\). \\
    In tal caso si dice che \(\Psi\) è autovettore di \(A\) relativo ad \(\hat{a}\) .
    Lo spettro discreto si definisce come:
    \[
        \sigma_d(A)  = \left\{ \hat{a} \in \mathbb{R} : \exists\, \Psi \in D_A, \Psi \neq 0 , (A-\hat{a})\Psi =0\right\}
    \]
\end{definition}

\begin{theorem}
    \(\Sigma\) autostato di \(\mathcal{A}\) relativo ad \(\hat{a} \in \mathbb{R}\) sse \(\Psi\) autovettore di \(A\) relativo ad \(\hat{a} \in \mathbb{R}\) .
\end{theorem}
\begin{proof}
    \[
        (\Delta \mathcal{A})^2_{\hat{a},\Sigma}= \frac{\expval{(A-\hat{a}\mathbb{I})^2}{\Psi}}{\norm{\Psi}^2}= 
        \frac{\braket{(A-\hat{a}\mathbb{I})\Psi}{(A-\hat{a}\mathbb{I})\Psi}}{\norm{\Psi}^2} = \frac{\norm{(A-\hat{a}\mathbb{I})\Psi}^2}{\norm{\Psi}^2}
    \]
    \[
        (\Delta\mathcal{A})_{\hat{a},\Sigma} = \iff (A-\hat{a}\mathbb{I})\Psi=0 
    \]    
\end{proof}

\begin{remark}
    Vale dunque : \(\sigma_d(\mathcal{A})= \sigma_d(A)\) .
\end{remark}

\begin{remark}
    Se \(\Psi_1\) e \(\Psi_2\) sono autovettori relativi ad \(\hat{a} \in \sigma_d(A)\), allora \(\alpha\Psi_1+ \beta\Psi_2\) è autovettore relativo ad \(\hat{a}\) .
\end{remark}
\begin{definition}
    Per \(\forall a \in \sigma_d(A) , \; \exists\) autospazio \(\mathcal{H}_a\subseteq\mathcal{H}\) di autovettori, 
    la cui dimensione \(d(a):= \operatorname{dim}\mathcal{A}_a\) è detta \textit{degenerazione} di \(a in \sigma_d(A)\) .\\
    Se \(d(a)=1,\; a \) è detto \textit{non-degenere} e \(\Psi\in \mathcal{H}_a\) è unico a meno riscalamenti. Se \(d(a)>1, \; a\) è detto \textit{degenere}.
\end{definition}

\begin{theorem}
    Se \(A\) è hermitiano, dati due autovalori \(a_1, a_2 \in \sigma_d(A)\) con \(a_1\neq a_2\) e i rispettivi autovettori 
    \(\Psi_1 \in \mathcal{H}_{a_1}, \Psi_2 \in \mathcal{H}_{a_2}\), allora \(\braket{\Psi_1}{\Psi_2}=0\).\\
    Cioè, autovettori relativi ad autovalori diversi sono ortogonali tra loro.
\end{theorem}
\begin{proof}
    Poiché \(A\) è hermitiano, vale \(\braket{\Psi_1}{A\Psi_2}= \braket{A\Psi_1}{\Psi_2}\). Essendo 
    \(A\Psi_1=a_1\Psi_1\) e \(A\Psi_2=a_2\Psi_2\), otteniamo
    \[
        \braket{\Psi_1}{A\Psi_2}= a_2\braket{\Psi_1}{\Psi_2} \qquad \text{e} \qquad
        \braket{A\Psi_1}{\Psi_2}= a_1\braket{\Psi_1}{\Psi_2}.
    \]
    Sottraendo membro a membro segue \((a_2-a_1)\braket{\Psi_1}{\Psi_2}=0\). Poiché \(a_1\neq a_2\), 
    necessariamente \(\braket{\Psi_1}{\Psi_2}=0\).
\end{proof}

\begin{remark}
    Se \(\dim \mathcal{H} < +\infty\), il numero di autovalori distinti di \(A\) è finito e al più uguale a \(\dim \mathcal{H}\).
    Se \(\mathcal{H}\) è separabile, allora \(\sigma_d(A)\) è al più numerabile.\\
    Ne segue che lo spettro discreto è sempre un insieme finito oppure, se infinito, comunque numerabile.
\end{remark}

\begin{theorem}
    {\normalfont\textbf{Decomposizione spettrale.}}
    Se \(A\) è operatore autoaggiunto, allora \(\mathcal{H}\) ammette base ortonormale di autovettori:
    \begin{equation}
        \left\{ \ket{a, r} \right\}_{\substack{a \in \sigma_d(A)\\ r = 1,\dots,d(a)}}
        \qquad
        \mathcal{H} = \bigoplus_{a \in \sigma_d(A)} \mathcal{H}_a.
    \end{equation}
\end{theorem}

Possiamo quindi scomporre ogni vettore nella forma:
\[
    \ket{\Psi} = \sum_{a \in \sigma_d(A)}  \sum_{r=1}^{d(a)} \ket{a,r}\braket{a,r}{\Psi}= \sum_{a \in \sigma_d(A)}\sum_{r=1}^{d(a)}    C_{a,r} \ket{a,r}    
\]
e ottenere:
\[
    \expval{\mathcal{A}}_{\Sigma}= \sum_{a \in \sigma(\mathcal{A})} a P_{\Sigma}(a) \quad \expval{A}_{\Psi} = \frac{\expval{A}{\Psi}}{\norm{\Psi}^2} =
    \sum_{a \in \sigma_d(A)}\sum_{r=1}^{d(a)}\frac{\bra{\Psi}A \ket{a,r}\bra{a,r}\ket{\Psi}}{\norm{\Psi}^2} = 
\]
\[
    = \sum_{a \in \sigma_d(A)} \sum_{r=1}^{d(a)}\frac{a \braket{\Psi}{a,r}\braket{a,r}{\Psi}}{\norm{\Psi}^2}=
    \sum_{a \in \sigma_d(A)} a \sum_{r=1}^{d(a)} \frac{\abs{\braket{a,r}{\Psi}}^2}{\norm{\Psi}^2}
    =\sum_{a \in \sigma_d(A)} a \sum_{r=1}^{d(a)} \frac{\abs{C_{a,r}}^2}{\norm{\Psi}^2}
\]
\[
    \expval{\mathcal{A}}_\Sigma = \expval{A}_\Psi \ \forall\,\Sigma  \implies P_\Sigma(a)   = \sum_{r=1}^{d(a)} \frac{\abs{\braket{a,r}{\Psi}}^2}{\norm{\Psi}^2}=
    \sum_{r=1}^{d(a)} \frac{\abs{C_{a,r}}^2}{\norm{\Psi}^2}
\]
Nel caso non degenere (\(d(a)=1\))  questa equazione si riconduce alle legge di Born: 
\[
    P_\Sigma(a)= \frac{\abs{\braket{a}{\Psi}}^2}{\norm{\Psi}^2}
\]

Rispetto a questa base, la rappresentazione matriciale è diagonale:
\[
A =
\begin{pmatrix}
\begin{matrix}
    a_1& & \\
    & \ddots &\\
    & & a_1
\end{matrix} &\mathbb{O}\\
\mathbb{O} & \begin{matrix}
    a_2& & \\
    & \ddots &\\
    & & \ddots
\end{matrix} 
\end{pmatrix},
\qquad
\matrixel{a',r'}{A}{a,r} = a\,\delta_{a'a}\,\delta_{r'r}.
\]

\subsection{Spettri continui}

Nel caso di \(\dim \mathcal{H}= \infty\) non è detto che esista base ortonormale di autovettori per \(A\),
se esiste allora abbiamo ancora uno spettro discreto e vale quanto detto prima, in caso contrario abbiamo uno spettro continuo.

\begin{example}
    Consideriamo, come al solito, una particella senza spin in 1 dim infinita. Esiste uno stato \(\Sigma\) con una posizione ben definita?\\
    \(\forall\, x_0 \in \mathbb{R}, \ (\Delta \mathcal{X})_{x_0, \Sigma}\) può essere arbitrariamente piccolo, ma non potrà mai essere uguale a \(0\).
\end{example}

Cerchiamo le autovettore dell'operatore posizione \(X\) relativi a un possibile autovalore \(x_0\):
\[
    (X-x_0)\Psi(x)= 0 \iff (x-x_0)\Psi(x)= 0 \ \forall x \implies \Psi(x) = 0 \ \forall x \neq x_0
\]
Ma \(\Psi= 0\) non è autovettore , quindi nessun \(x_0 \in \mathbb{R}\) è autovalore e nessuna \(\Psi \in D_X\subseteq L_2(\mathbb{R},\dd{x})\) è autofunzione.

Lo stesso vale per l'operatore momento \(P\):
\[
    (P-p)\Psi(x)=0 \iff ( -i\hbar\dv{x} - p ) \Psi(x) = \iff \Psi(x)= (cost)e^{i\frac{p}{\hbar}x} \notin L_2(\mathbb{R})
\]

Per trovare soluzioni dobbiamo estendere il dominio da \(L_2(\mathbb{R},\dd{x})\) allo spazio delle ditribuzioni \(S^*(\mathbb{R})\), 
dunque l'operatore posizione ora sarà: 
\[
    X: S^*(\mathbb{R})\to S^*(\mathbb{R}) \ \ \phi(x)\mapsto x\phi(x)
\]
Risolviamo ora l'equazioni agli autovalori per \(\Psi(x) \in S^*(\mathbb{R}), \Psi\neq 0 \):
\begin{equation}
    (X-x_0)\Psi(x)= 0 \iff \Psi(x)= \delta(x-x_0) = : \ket{x_0} \in S^*(\mathbb{R})
\end{equation}

La soluzione esiste \(\forall\, x_0 \in \mathbb{R}\) ed è unica a meno di riscalamenti dunque \(x_0\) è non degenere.
Chiamiamo gli \(x_0 \in \mathbb{R}\) \textit{autovalori generalizzati} di \(X\) e  i \(\ket{x_0}\)  \textit{autovettori generalizzati}. 
L'insieme degli autovettori \(\left\{ \ket{x_0} \right\}_{x_0 \in \mathbb{R}}\) è una \textit{base generalizzata}. Dunque posso scomporre ogni stato come:
\[
    \ket{\Psi}= \int_{\mathbb{R}}\dd{x_0} \ket{x_0} \braket{x_0}{\Psi} = \int_\mathbb{R} \dd{x_0} \Psi(x_0) \ket{x_0} 
    \quad \rho_\Sigma(x_0) = \frac{\abs{\Psi(x_0)}^2}{\norm{\Psi}^2}
\]
Stessa cosa vale per il momento:
\[
    P: S^*(\mathbb{R})\to S^*(\mathbb{R}),\ \  \Psi(x) \mapsto -i\hbar \dv{x}\Psi(x)
\]
\[
    \ket{p}:= \frac{1}{\sqrt{2\pi \hbar}}e ^{i\frac{p}{\hbar}x} \text{ è soluzione di }  (P-p)\ket{p}=0
\]
\[
    \ket{\Psi} = \int_\mathbb{R} \dd{p} \ket{p} \braket{p}{\Psi}= \int_\mathbb{R} \dd{p} \tilde{\Psi}(p)\ket{p} \quad 
    \rho_\Sigma(p)= \frac{\abs{\tilde{\Psi}(p)}^2}{\norm{\Psi}^2}
\]

\begin{remark}
    La normalizzazione di \(\ket{x}\) e \(\ket{p}\) è fissata dalla condizione:
    \begin{equation}
        \int \dd{x} \ket{x}\bra{x} = \mathbb{I} = \int \dd{p} \ket{p}\bra{p}
    \end{equation}

\end{remark}

In generale dato uno spazio di Hilbert di dimensione infinita \(\mathcal{H}= L_2(\mathbb{R}^n)\) e un operatore autoaggiunto \(A: D_A \subseteq L_2(\mathbb{R}^n)\to L_2(\mathbb{R}^n)\)
si estende a:
\begin{equation}
    A: S^*(\mathbb{R}^n)\to S^*(\mathbb{R}^n) ; \ \phi \mapsto A\phi
\end{equation}
\(A\phi \in S^*(\mathbb{R}^n)\) è definito da:
\[
    \braket{A\phi}{\psi}:=\braket{\phi}{A\psi} \quad \forall\, \psi \in S(\mathbb{R}^n)
\]
\begin{attention}
    L'estensione è sensata se \(S(\mathbb{R}^n)\subseteq D_A\), se \(\mathcal{H}= L_2(X, \dd{\mu}), \, X \subseteq \mathbb{R}^n \) non è ovvio 
    quale sia il corretto spazio \(S^*\) di distribuzioni.
\end{attention}
\begin{definition}
    \(a \in \mathbb{R}\) è \textit{autovalore generalizzato} di \(A\) sse:
    \[
        \exists\Psi \in S^*(\mathbb{R}^n), \,\Psi\neq 0 : \ (A-a)\Psi= 0
    \]
    In tal caso, \(\Psi\) è \textit{autovettore generalizzato}.
\end{definition}

Si dimostra che \(\sigma(A) = \left\{ \text{ autovalori generalizzati } \right\}\) .
\begin{definition}
    \(a \in \sigma(a)\) è \textit{autovalore proprio} se il suo autovettore \(\Psi \in D_a\subseteq L_2(\mathbb{R}^n)\)
\end{definition}

Si dimostra che \(\sigma_d(A)= \left\{ \text{ autovalori propri } \right\} \subseteq \sigma(A)\)

\begin{theorem}
    {\normalfont \textbf{Decomposizione spettrale generalizzato.}}\\
    Dato un operatore \(A\) densamente definito e autoaggiunto, allora esiste un base di autovettori di \(A\) (propri o generalizzati):
    \(
        \left\{ \ket{a, r} \right\}_{\substack{a \in \sigma(A)\\ r = 1,\dots,d(a)}}
    \) tale che \(\forall \ket{\Psi} \in \mathcal{H}\) vale:
    \begin{equation}
        \ket{\Psi}= \sum_{a \in \sigma_d(A)} \sum_{r=1}^{d(a)} \ket{a,r}\overset{C_{a,r}}{\braket{a,r}{\Psi}} +
        \int_{\sigma_c(A)}\sum_{r=1}^{d(A)}\ket{a,r}\overset{\Psi_r(a)}{\braket{a,r}{\Psi}}\dd{a}
    \end{equation}
    \begin{equation}
        \norm{\Psi}^2=\sum_{a \in \sigma_d(A)} \sum_{r=1}^{d(a)} \abs{\braket{a,r}{\Psi}}^2 +
        \int_{\sigma_c(A)}\sum_{r=1}^{d(A)}\abs{\braket{a,r}{\Psi}}^2\dd{a}
    \end{equation}

\end{theorem}

\begin{attention}
    Il teorema non vale se \(A\) è hermitiano, ma non autoaggiunto.
\end{attention}

