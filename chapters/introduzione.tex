% 1 Lezione del 29/09/2025


\chapter{Introduzione}


Iniziamo trattando alcune caratteristiche della meccanica quantistica, 
che la differenziano dalla meccanica classica: \textbf{linearità}, presenza di \textbf{numeri complessi} e \textbf{perdita del determinismo}.


\subsection*{1 Linearità}

Descriviamo sistemi fisici attraverso variabili dinamiche \(u(t,\dots)\) che soddisfano le equazioni del moto.
\begin{definition}
    Abbiamo una \textit{teoria lineare} se le equazioni del moto hanno forma \(L u=0\), con \(L \) \textit{operatore lineare}, 
    cioè se vale:
    \begin{itemize}
        \item \(L(u_1+u_2)= L(u_1)+L(u_2)\)
        \item \(L(\alpha u)= \alpha L (u)\)
    \end{itemize}
\end{definition}

\begin{remark}
    La sovrapposizione lineare di soluzioni è ancora soluzione, infatti
    se \(u_1,u_2\) sono soluzioni, cioè \(L(u_1)=0\) e \(L(u_2)=0\), allora \(\alpha u_1+\beta u_2\) è soluzione dato che vale:
    \[L(\alpha u_1+\beta u_2)= \alpha L(u_1)+ \beta L(u_2)=0
    \] 
\end{remark}
\begin{remark}
    Se la teoria è lineare, le soluzioni formano uno spazio vettoriale.
\end{remark}    

\begin{example}
    Sono equazioni lineari:
    \begin{itemize}
        \item Equazione di Schrödinger: \(i\hbar \dv{t}\Psi= H \Psi\)
        \item Equazioni di Maxwell.
        \item Equazioni delle onde.
    \end{itemize}
\end{example}
\begin{example}
    Non sono equazioni lineari:
    \begin{itemize}
        \item Le equazioni di Eisntein in relatività generale.
        \item Le equazioni della dinamica di Newton.
    \end{itemize}
\end{example}


\subsection*{2 Numeri complessi}

In meccanica classica, i numeri complessi possono essere utili formalmente, ma non sono indispensabili;
tutte le variabili dinamiche e le osservabili sono numeri reali.\\
In meccanica quantistica, le soluzioni \(\Psi(x,t)\) all'equazione di Schrödinger sono necessariamente complesse. 
Tuttavia, le osservabili devono essere reali e dipenderanno dunque indirettamente dalla funzione d'onda, ad esempio \(\abs{\Psi(x,t)}^2\).

\begin{remark}
    In linea di principio è possibile formulare la meccanica quantistica senza l'utilizzo di numeri complessi,
    ma sarebbe tremendamente complicato e alcune proprietà come la linearità non sarebbero più evidenti.
\end{remark}


\subsection*{3 Assenza di determinismo}
A differenza della meccanica classica, la meccanica quantistica non è deterministica, ma fornisce soltanto una descrizione probabilistica.
\begin{example}
    Consideriamo un fascio di luce, polarizzata linearmente a un angolo \(\alpha\) rispetto a un arbitrario asse \(x\),
    che attraversa un filtro polarizzatore lungo x.
    \begin{figure}[htbp]
        \centering
        % Figura TikZ: filtro polarizzatore lungo x e campo E inclinato di alpha
        \begin{tikzpicture}[scale=1.05]
            % parametro: angolo alpha (in gradi)
            \def\ang{35}

            % Assi
            \draw[->] (-3.8,0) -- (4.2,0) node[below right] {$x$};
            \draw[->] (0,-0.6) -- (0,3.2) node[left] {$y$};

            % Filtro polarizzatore orientato lungo l'asse x (capsula simmetrica)
            \draw[rounded corners=2pt, line width=0.9pt, color = blue]
                (-3.0,-0.2) rectangle (3.0,0.2);

            % Vettore del campo elettrico incidente a angolo alpha
            \draw[->, thick]
                (0,0) -- ({2.8*cos(\ang)},{2.8*sin(\ang)})
                node[above right] {$\vec{E}_{\mathrm{in}}$};

            % Arco che indica l'angolo alpha rispetto a x
            \draw[thick,->] (1.0,0) arc (0:\ang:1.0);
            \node at ({1.15*cos(0.5*\ang)},{1.15*sin(0.5*\ang)}) {$\alpha$};
        \end{tikzpicture}
        \caption{Filtro polarizzatore e onda incidente.}
        \label{fig:polarizzatore}
    \end{figure}

    Sappiamo che il rapporto tra l'intensità luminosa in uscita dal filtro e quella in ingresso è 
    data dalla \textbf{Legge di Malus}: 
    \[
    \frac{I_{out}}{I_{in}}=\cos[2](\alpha)
    \]
    Inoltre, la frequenza del fascio rimane invariata; dunque, sapendo che l'energia di ogni fotone è
    \(\varepsilon= h \nu\), possiamo interpretare \(\cos[2](\alpha)\) come la probabilità che un singolo fotone
    attraversi il filtro.
\end{example}
